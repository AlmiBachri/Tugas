\chapter*{Teori}

\begin{enumerate}
	\item Fungsi adalah sub-program yang dapat digunakan kembali baik didalam program tersebut maupun di program lainnya, fungsi didalam python adalah def\\
	inputan fungsi adalah untuk mengambil suatu data\\
	contoh kode program fungsi:\\
	def nama():\\
	  print "pute cute"\\
	 \#memanggil fungsi\\ 
	 nama()
	
	\item paket pada python adalah modul yang berisikan kode-kode python dan dapat di impor kedalam program\\
	cara memanggil : from nama\_package import library
	
	\item kelas adalah cara membuat kode supaya lebih mudah mengorganisasikan berbagai fungsi\\
	contoh kode program: class\_nama\_orang\\
	objek adalah sesuatu yang dihasilkan oleh kelas\\
	atribut adalah variabel yang dapat dideklarasikan\\
	method adalah kode yang berisikan sebuah fungsi dari suatu kelas\\
	contoh program: \\
	def namaMethod()\\
	perintah yang akan dijalankan\\
	return nilai
	
	\item pemanggilan library pada sebuah folder pute:\\
	def hai():\\
	print("Selamat Pagi")\\
	tetapi sebelumnya harus mengimprot terlebih dahulu\\
	import nov 
	nov.hai()
	
	\item perintah form kalkulator penambahan terdapat pada file (no.5\_kalkulator)seperti:\\
	from no.5\_kalkulator import penjumlahan
	
	\item Cara memanggil paket fungsi file library di dalam folder harus menuliskan foldernya terlebih dahulu lalu menginport nama librarynya\\
	from bernyanyi import lagu 
	
	\item Cara memanggil paket paket fungsi kelas library di dalam folder\\
	from bernyayi import lagu\\
	nama kelas yang akan digunakan adalah lagu
	


\end{enumerate}
