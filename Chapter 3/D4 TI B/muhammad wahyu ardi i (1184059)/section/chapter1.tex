\chapter{Fungsi dan Kelas}

\section{Teori}
\subsection{Fungsi}
\begin{enumerate}
\item Fungsi 
\par 
Fungsi yaitu sebuah blok kode yang akan dieksekusi ketika dipanggil dalam suatu program
\item Parameter 
\par 
Parameter yaitu inputan sebuah fungsi bertujuan sebagai menyimpan sebuah nilai
\item return 
\par 
Digunakan untuk mengembalikan sebuah nilai, bisa juga untuk mengakhiri eksekusi sebuah fungsi
\begin{lstlisting}[language=Python]
def fungsi(a,b):
	c=a+b
	return c
\end{lstlisting}
\end{enumerate}

\subsection{Package}
\begin{enumerate}
\item Package
\par 
package yaitu sebuah folder yang menyimpan source code, misalnya pada tempat kita menyimpan main program, kita membuat sebuah folder motor dan di dalamnya kita membuat sebuah source code dengan nama mesin.py.

\end{enumerate}

\subsection{Class}
\begin{enumerate}
\item Class
\par 
class merupakan blueprint dari sebuah object, jika diibaratkan membuat sebuah kue, class merupakan cetakan kuenya, contoh kode :
\begin{lstlisting}[language=Python]
class Nama:
    def __init__(self,nama):
        self.nama = nama
    def helonama(self):
        print("Helo",nama)
\end{lstlisting}

\item Object
\par 
object merupakan hasil cetakan dari sebuah class contoh kode :
\begin{lstlisting}[language=Python]
#import kelas terlebih dahulu
import kelas3lib
#membuat object
cobakelas=kelas3lib.Kelas3ngitung(npm) 
hasilkelas=cobakelas.npm1()
\end{lstlisting}

\item Atrribute
\par 
attribute merupakan variabel global yang dimiliki oleh sebuah class
\begin{lstlisting}[language=Python]
class Kelas3ngitung:
	#pendefinisian attribute
    def __init__(self,nama):
        self.nama = nama
\end{lstlisting}

\item Method
\par 
method merupakan fungsi-fungsi dalam sebuah class
\begin{lstlisting}[language=Python]
class Nama:
    def __init__(self,nama):
        self.nama = nama
    #Pembuatan method pada class
    def nama(self):
       	print("hello",nama,",apa kabar ?")
\end{lstlisting}

\begin{lstlisting}[language=Python]
import kelas3lib
cobakelas=nama.Nama(nama) 
#pemanggilan method pada program
hasilkelas=cobakelas.nama()
\end{lstlisting}
\end{enumerate}

\subsection{Penggunaan library}
\par 
contoh membuat sebuah library, contoh disini kita membuat pada folder libra :
\begin{lstlisting}{language=Python}
def helo():
    print("Hello world")
\end{lstlisting}
contoh jika kita ingin memanggil fungsi dari library pada main program kita harus terlebih dahulu import :
\begin{lstlisting}{language=Python}
#import library yang telah dibuat
import libra
#pemanggilan fungsi pada library
libra.helo()
\end{lstlisting}

\subsection{pemakaian package from kalkulator import penambahan}
\par 
\begin{lstlisting}{language=Python}
from kalkulator import penambahan
\end{lstlisting}
kode diatas yaitu program memanggil sebuah package terlebih dahulu, setelah itu menambahkan source code penambahan, kode diatas dapat dibaca seperti ini ”import penambahan dari folder kalkulator”, contoh lainnya :
\begin{lstlisting}{language=Python}
from dapur import memasak
\end{lstlisting}

\subsection{pemanggilan library dalam sebuah folder}
\par 
untuk mengakses sebuah library dalam sebuah folder kita perlu menuliskan foldernya terlebih dahulu setelah itu mengimport nama librarynya, contoh :
\begin{lstlisting}{language=Python}
from me import libheart
\end{lstlisting}
artinya dalam package me kita akan memakai library libheart

\subsection{pemanggilan class dalam sebuah folder}
\par 
untuk mengakses sebuah class dalam suatu folder, kita perlu menuliskan foldernya terlebih dahulu lalu mengimport nama class nya, contoh :
\begin{lstlisting}{language=Python}
from me import clheart
\end{lstlisting}
artinya dalam package me kita akan memakai class cheart

\chapter{Keterampilan pemrograman}
\section*{1. Soal 1}
\lstinputlisting[language=Python]{src/npm1.py}
\section*{2. Soal 2}
\lstinputlisting[language=Python]{src/npm2.py}
\section*{3. Soal 3}
\lstinputlisting[language=Python]{src/npm3.py}
\section*{4. Soal 4}
\lstinputlisting[language=Python]{src/npm4.py}
\section*{5. Soal 5}
\lstinputlisting[language=Python]{src/npm5.py}
\section*{6. Soal 6}
\lstinputlisting[language=Python]{src/npm6.py}
\section*{7. Soal 7}
\lstinputlisting[language=Python]{src/npm7.py}
\section*{8. Soal 8}
\lstinputlisting[language=Python]{src/npm8.py}
\section*{9. Soal 9}
\lstinputlisting[language=Python]{src/npm9.py}
\section*{10. Soal 10}
\lstinputlisting[language=Python]{src/npm10.py}
\section*{11. Soal 11}
\lstinputlisting[language=Python]{src/lib3.py}
\section*{12. Soal 12}
\lstinputlisting[language=Python]{src/kelas3lib.py}
\section*{13. main.py}
\lstinputlisting[language=Python]{src/main.py}


\section*{Keterampilan penanganan error}
\subsection{penanganan error}
error :\\
TypeError: \_\_init\_\_() missing 1 required positional argument: 'npm'\\
solusi :\\
menambahkan parameter pada fungsi
\section*{Try except}
\begin{lstlisting}{language=Python}
def pembagian(a,b):
    c=a/b
    return c

d=int(input("angka pertama : "))
e=int(input("angka kedua : "))
try:
    print(pembagian(d,e))
except:
    print("jangan masukan angka 0")
\end{lstlisting}