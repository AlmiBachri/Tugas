\section{Fungsi}
		\subsection{pemahaman teori}
			\begin{enumerate}
				\item fungsi\\
				Fungsi adalah blok kode program yang akan dijalankan ketika  sebuah
				program di panggil.\\
				
				\item parameter adalah variabel yang dapat menyimpan nilai untuk diproses didalam sebuah fungsi.
				
				\item return memiliki fungsi untuk mengembalikan nilai ataupun variabel.
				\begin{lstlisting}[language=Python]
				def fungsi(m,n):
						p=m+n
						return p
				\end{lstlisting}
				
				\item item paket adalah direktory yang berisi file python dan file dengan nama \_init\_.py. Jadi jika ada direktori didalam python dengan nama \_init\_.py, akan terdeteksi sebagai paket oleh python.Cara memanggil sebuah paket atau library adalah dengan \textit{import} nama paket atau library tersebut.
				\begin{lstlisting}[language=Python]
				from kamar import buku
				\end{lstlisting}
				
				\item class adalah sebuah blueprint dari sebuah objek yang akan di buat.
				\begin{lstlisting}[language=Python]
	class World:
	    def __init__(self,World):
	        self.World = World
	    def heloWorld(self):
	        print("Hello",World)
				\end{lstlisting}
				
				\item Jika class adalah blue print maka objek adalah hasil cetak dari sebuah kelas. objek memiliki variable dan kode yang saling terhubung. objek di buat dengan class.
				\begin{lstlisting}[language=Python]
	#import kelas terlebih dahulu
	import kelas3lib
	#membuat object
	cobakelas=kelas3lib.Kelas3ngitung(npm) 
	hasilkelas=cobakelas.npm4()
				\end{lstlisting}
				
				\item attribut adalah sebuah tempat untuk menampung data atau perintah yang berhubungan dengan attribut tersebut
				\begin{lstlisting}[language=Python]
	class Kelas3ngitung:
		#pendefinisian attribute
	    def __init__(self,World):
	        self.World = World
				\end{lstlisting}
				
				\item method adalah sebuah fungsi yand ada di dalam class.
				\begin{lstlisting}[language=Python]
	class world:
	    def __init__(self,world):
	        self.world = world
	    #Pembuatan method pada class
	    def world(self):
	       	print("hello",world,",apa kabar ?")
				\end{lstlisting}
				
				\item contoh membuat sebuah library,kita membuat pada folder library :
				\begin{lstlisting}{language=Python}
	def hello():
	    print("Hello world")
				\end{lstlisting}
	
	            \newpage
				\item contoh jika kita ingin memanggil fungsi dari library pada main program kita harus terlebih dahulu import :
				\begin{lstlisting}{language=Python}
	#import library yang telah dibuat
	import library
	#pemanggilan fungsi pada library
	library.hello()
				\end{lstlisting}
				
				\item pemakaian package from kamar import buku
				\begin{lstlisting}{language=Python}
	from kalkulator import penambahan
				\end{lstlisting}
	kode diatas berarti program memanggil sebuah package terlebih dahulu baru menambahkan source code buku, kode diatas dapat dibaca seperti ini "import buku dari folder kamar"
	
				\item pemanggilan library dalam sebuah folder\\
				untuk mengakses sebuah library dalam sebuah folder kita perlu menuliskan foldernya terlebih dahulu lalu mengimport nama librarynya, contoh :
				\begin{lstlisting}{language=Python}
	from mahasiswa import dasi
				\end{lstlisting}
	artinya dalam package mahasiswa akan memakai library dasi
	
				\item pemanggilan class dalam sebuah folder\\
				untuk mengakses sebuah class dalam sebuah folder kita perlu menuliskan foldernya dulu kemudian mengimport nama class nya, contoh :
				\begin{lstlisting}{language=Python}
	from mahasiswa import kampus
				\end{lstlisting}
	artinya dalam package mahasiswa kita akan memakai class kampus
				
				\end{enumerate}
	
	    \newpage			
		\subsection{Ketrampilan Pemrograman}
				\begin{enumerate}
					\item \lstinputlisting[language=Python]{src/npm1.py}
					\item \lstinputlisting[language=Python]{src/npm2.py}
					\item \lstinputlisting[language=Python]{src/npm3.py}
					\item \lstinputlisting[language=Python]{src/npm4.py}
					\item \lstinputlisting[language=Python]{src/npm5.py}
					\item \lstinputlisting[language=Python]{src/npm6.py}
					\item \lstinputlisting[language=Python]{src/npm7.py}
					\item \lstinputlisting[language=Python]{src/npm8.py}
					\item \lstinputlisting[language=Python]{src/npm9.py}
					\item \lstinputlisting[language=Python]{src/npm10.py}
					\item \lstinputlisting[language=Python]{src/lib3.py}
					\item \lstinputlisting[language=Python]{src/kelas3lib.py}
					\item \lstinputlisting[language=Python]{src/main.py}
		
	\end{enumerate}

